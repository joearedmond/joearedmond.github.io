\documentclass[11pt]{article}

% dependencies
\usepackage{hyperref} % hyper links
\usepackage{fontawesome} % icons
\usepackage{geometry} % page formatting
\usepackage{titlesec} % section style
\usepackage{enumitem} %itemize style

% formatting
\usepackage[utf8]{inputenc}
\pagenumbering{gobble} % remove page numbers
\geometry{letterpaper, margin=0.5in}
\setitemize[0]{leftmargin=*, topsep=0pt, itemsep=-0.5ex}
\hyphenpenalty=10000

% title style
\title{\vspace{-2.7em}\LARGE\textbf{
    Joe Redmond \\
    }\large
    New York, NY
    \vspace{-4.5em}}
\date{}
\author{}

% section style via titlesec
\titleformat{\section}[hang]{\bfseries\Large}{}{0pt}{}[\hrule] % hline after section title
\titlespacing*{\section}{0pt}{5pt}{5pt} % spacing before and after section title

% Resume Entry tools
\newcommand{\bfentry}[2]{\vspace{1mm}\textbf{#1} \hfill \textbf{#2}}
\newcommand{\itentry}[2]{\vspace{0.5mm}\textit{#1} \hfill \textit{#2}}
\newcommand{\bfitentry}[2]{\vspace{0.5mm}\textbf{#1} \hfill \textit{#2}}
\newcommand{\bfitem}[2]{\textbf{#1} \quad #2}
\newcommand{\myproject}[2]{\vspace{1mm}\textbf{#1} \hfill \faGithub \hspace{1mm} \href{#2}{#2}}


\begin{document}

%\setlength{\droptitle}{-11em}
\maketitle
\begin{center}
    \begin{tabular}{ c c c c }
        \href{https://joearedmond.github.io/}{
            \faGlobe \hspace{1mm} joearedmond.github.io
        } &
        \href{mailto:joe.redmond@columbia.edu}{
            \faEnvelope \hspace{1mm} joe.redmond@columbia.edu
        } &
            \faPhone \hspace{1mm} (303) 374-4087 
        &
        \href{https://www.linkedin.com/in/joearedmond/}{
            \faLinkedinSquare \hspace{1mm} joearedmond
        } \\  
    \end{tabular}
\end{center}
\vspace{-1.5em}

\section{Education}
\bfentry{Columbia University}{New York, NY}\\
\itentry{MS, Computer Science, Machine Learning. GPA 3.5}{Sept 2020 -- July 2022} \vspace{0.4em}\\
\bfentry{Princeton University}{Princeton, NJ}\\
\itentry{BSE, Biological Engineering (CBE). Minor in Computer Science (PAC). GPA 3.2}{Aug 2014 -- May 2018}

\section{Skills}
\bfitem{Languages}{Python, Go, R, C, Java, SQL, Matlab} \\
\bfitem{Packages}{Python: PyTorch, sklearn, NumPy. R: tidyverse, ggplot2, plumber. C: sockets API, kernel libs} \\
\bfitem{Tools}{GCP, git, vim, SQL databases (IBM, Microsoft), LaTeX, Adobe Creative Suite}

\section{Professional Experience}
\bfentry{Memorial Sloan Kettering Cancer Center}{New York, NY} \\
\itentry{Data Scientist}{June 2021 (Current)}
\begin{itemize}
    \item Built near-real-time pipeline to serve random forest regression predictions into Electronic Health Record using R (plumber), Linux, and IBM DB2 to realize operating room scheduling accuracy improvement of 17\%.
    \item Led adaptation of HL7 FHIR health data standard for 6 developers across 4 teams for above application to ensure robust pipeline integration. Wrote serialization specification to integrate model accurately.
    \item Added in-memory caching and error recovery logic to application, improving time-to-prediction by 83\% and reliability by 4\%. Designed telemetry system to monitor system performance.
    \item Led development for company-wide R package interfacing with vendor APIs to help analysts to perform I/O operations, notably with Microsoft Teams via MS Graph API. Managed issues and changes from 6 developers.
\end{itemize}
\itentry{Data Analyst}{Sept 2019 -- June 2021}
\begin{itemize}
    \item Deployed infrastructure within 2 days for what is now most-viewed Covid-19 dashboard in hospital (>4000 views). Project helped leadership know when to re-open operating rooms while protecting ICU utilization.
    \item Designed separation of front and back end for Covid-19 dashboard, allowing team to work in parallel. Built back end in R, bash, and SQL consisting of 4 web scrapes, 2 database ETLs, and 3 web API data feeds.
\end{itemize}
\vspace{1mm}
\bfentry{Precision Xtract}{Stamford, CT} \\
\vspace{0.5mm}\textit{Analyst:} Introduced web scraping framework to generate physician-hospital crosswalk. \hfill \textit{Sept 2018 -- Sept 2019}

\section{Academic Experience}
\bfitem{Courses}{Deep Learning Systems, Distributed Systems, Algorithms, OS, Machine Learning Theory} \vspace{1mm} \\ % weird spacing stuff, not sure sorry
\bfitentry{Teaching Assistant, Artificial Intelligence}{Fall 2021}
\begin{itemize}
    \item Hosted office hours and graded student work for masters-level survey course, including search (heuristic, adversarial, backtracking) and machine learning (SVMs, decision trees). Student implementations in Python.
\end{itemize}
\myproject{Distributed Key Value Database (Go)}{https://tinyurl.com/yckrascc}
\begin{itemize}
    \item Designed a distributed key-value storage database that includes both sharded load balancing and shard replication.
    \item Used Paxos consensus algorithm to implement replicated write-ahead log to ensure serializable consistency.
\end{itemize}
\myproject{From-Scratch Neural Network Image Reconstruction (Python)}{https://git.io/JcvP8}
\begin{itemize}
    \item Wrote vectorized feed-forward neural network in Numpy to reconstruct images, using Adaptive Momentum (Adam) gradient descent to train weight and bias parameters.
    %\item Derived backpropagation algorithm and implementation from first principles in literature to achieve $O(n)$ runtime complexity in number of layers.
\end{itemize}

% removed project
\iffalse
\myproject{Weighted Round-Robin CPU Task Scheduler}{https://git.io/JcvSs}
\begin{itemize}
    \item Modified Linux kernel v5.4 to replace default multi-core CPU task scheduling algorithm with a Weighted Round-Robin algorithm, allowing users to modify how long a task is on the CPU via system call using C.
    %\item Exposed system call to modify the weight describing tasks' relative CPU runtime before comtext-switching.
    %\item Generalized approach to operate on multi-core architectures through task load balancing.
\end{itemize}
\fi

\section{Activities}
President, Princeton Triangle Club, a 125-year-old, 65-member collegiate musical comedy troupe (2017 -- 2018)

\end{document}

